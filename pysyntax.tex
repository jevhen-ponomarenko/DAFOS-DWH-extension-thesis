%%%%------------------------------------------------------------------------%%%%
%%%%           Semestrální práce z předmětu BI-TEX (pysyntax.tex)           %%%%
%%%%------------------------------------------------------------------------%%%%
%%%%
%%%% Zvýraznění syntaxe programovacího jazyka Python za pomoci makra OPmac
%%%% a použití existujících zvýraznění publikovaných na OPmac-Tricks.
%%%%
%%%% Autor: Petr Krajnik, krajnpet@fit.cvut.cz, LS2016
%%%% Historie hlavních změn viz konec tohoto zdrojového souboru.
%%%%
%%%% TODO: Nejspíš už nic. :-)
%%%%
\ifx\PySyntaxVersion\undefined\else \endinput \fi
\def\PySyntaxVersion {Apr.30 2016}
\immediate\write16 {This is PySyntax (Python synt. hightlighting for OPmac),
	version <\PySyntaxVersion>}

%%%% Definice zpráv pro toto makro.
\def\PySyntaxMessage #1#2{%
	\immediate\write16 {l.\the\inputlineno\space PySyntax #1: #2}%
}
\def\PySyntaxError   #1{\PySyntaxMessage{ERROR}{#1}\expandafter\end}
\def\PySyntaxWarning #1{\PySyntaxMessage{WARNING}{#1}}

%%%% Kontroly pro OPmac (vložení a verze).
\ifx\OPmacversion\undefined
	\PySyntaxError {For this macro OPmac must be loaded first.
		Please insert "\string\input opmac" before that line.}%
\fi
\ifx\ptthook\undefined
	\PySyntaxError {Loaded OPmac version must be later or equal
		than version <Oct. 2015>. Implementing "\string\ptthook".}%
\fi

%%%%%%%%%%%%%%%%%%%%%%%%%%%%%%%%%%%%%%%%%%%%%%%%%%%%%%%%%%%%%%%%%%%%%%%%%%%%%%%%

%%%% Optimalizovaná verze makra pro hledání a nahrazení retězce.
\bgroup\catcode`\!=3\catcode`\?=3
	%% Vyhledání a nahrazení vzorku v obsahu makra \tmpb.
	\gdef\replacestringx #1#2{%
		\long\def\findpatternA ##1#1##2#1{\ifx !##2\relax \addto\tmpb{##1}\else
			\addto\tmpb{##1#2##2}\expandafter\findpattern\fi}%
		\long\def\findpattern ##1#1{\ifx !##1\relax\else
			\addto\tmpb{#2##1}\expandafter\findpattern\fi}%
		\edef\tmpb{\expandafter}\expandafter\findpatternA\tmpb ?#1!#1%
		\expandafter\removeEoS\tmpb
	}
	%% Vyhledání vzorku ohraničeným vzorky #1<OBSAH>#2 a nahradí ji #3 v \tmpb.
	\gdef\replacerange #1#2#3{%
		\long\def\findpattern ##1#1##2#2{\ifx!##2\addto\tmpb{##1}\else
			\addto\tmpb{##1#3}\expandafter\findpattern\fi}%
		\edef\tmpb{\expandafter}\expandafter\findpattern\tmpb ?#1!#2%
		\expandafter\removeEoS\tmpb
	}
	%%%% Odstranění konce znaku řetězu. (EoS=EndOfString)
	\gdef\removeEoS #1?{\def\tmpb{#1}}
\egroup

%%%% Změna makra provedení vícenásobného nahrazení.
\def\doreplace #1#2 from #3{%
	\def\replace ##1{\ifx^##1^\else
		\replacestringx{#1}{#2}\expandafter\replace\fi}%
	\expandafter\replace #3{}%
}

%%%% Zapnutí lokální vs. globální zvýraznění syntaxe verbatim.
\def\hisyntax  #1{\bgroup \addto\tthook{\obeytabs}%
	\def\ptthook{\csname hisyntax#1\endcsname}}
\def\ghisyntax #1{\addto\tthook{\obeytabs}%
	\def\ppthook{\bgroup\csname hisyntax#1\endcsname}}

%%%% Zpracování odstazení tabulátory.
\newcount\tabSpaceCount \tabSpaceCount=4
\def\tabSkip {\ifnum\tabSpaceCount>0%
	\hskip \tabSpaceCount\fontdimen2\tentt\fi}
\def\obeytabs{\catcode`\^^I=\active}
\bgroup %% Chceme, aby v případě použití "nerozbil" makro begtt.
	\obeytabs\global\let^^I=\space
\egroup

%%% Zvýraznění syntaxe jazyka Python.
\def\hisyntaxPy #1\egroup{%
	%% Označení mezer, tabulátorů a konců řádků...
	\letMarksRelax
	\adef{ }{\n\ \n}\adef\^^I{\n\INDENT\n}\adef\^^M{\n\NLh\n}%
	\edef\tmpb{#1\egroup}%
	%% Označkování elementů kódu Pythonu...
	\markEOF
	\markStringLitPy
	\markCommentsPy
	\markOperatorsPy
	\markKeywordsPy
	\markNumbersPy
	\markStrPrefixPy
	\markDecorators
	%% Sazba označkovaného kódu...
	\let\INDENT=\tabSkip
	\def\O ##1{{\opStylePy ##1}}\let\STR=\PySyntaxFormat \let\U=\PySyntaxFormat
	\let\n=\empty \let\NLh=\par
	\localcolor\identStylePy \tentt\tmpb\egroup
}
\def\letMarksRelax {%
	\let\U=\relax \let\O=\relax \let\STR=\relax \let\INDENT=\relax
	\let\n=\relax \let\NLh=\relax
}
%%%% Formát číslování řádků.
\def\printttline{\llap{\sevenrm\Black\the\ttline\kern.9em}}

%%%% Makra pro označkování zdrojáku.
\def\markEOF {\replacestringx{\n\egroup}{\egroup}}
\def\markCommentsPy {%
	\expandafter\replacerange
		\expandafter{\string##}{\NLh}{\U{commentStylePy}{\###2}\NLh}%
}
\def\markOperatorsPy {%
	\doreplace{##1}{\n\O##1\n} from \opListPy
}
\def\markKeywordsPy {%
	\doreplace{\n##1\n}{\U{kwStylePy}{##1}} from \kwListPy
	\replacerange{\n class\n}{\n\O(\n}%
		{\U{kwStylePy}{class}\U{clsNameStylePy}{##2}\n\O(\n}%
	\replacerange{\n def\n}{\n\O(\n}%
		{\U{kwStylePy}{def}\U{defNameStylePy}{##2}\n\O(\n}%
}
\def\markDecorators {%
	\edef\tmpcmd {\noexpand\replacerange{%
		\n\O\string@\n}{\NLh}{\U{decorStylePy}{\string@####2}\NLh}}
	\tmpcmd
}

%%%% Zanačkování řetězcových literálů.
\def\markStringLitPy {\removeStrEscapes
	%% Uvozovací značky literálu.
	\replacestringx{"""}{\nextStrMark0{"""}}%
	\replacestringx{'''}{\nextStrMark1{'''}}%
	\replacestringx{"}{\nextStrMark2"}%
	\replacestringx{'}{\nextStrMark3'}%
	%% Označkování...
	\edef\tmpb{\tmpb}%
}
\edef\removeStrEscapes {%
	\noexpand\replacestringx{\string\"}{{\string\"}}%
	\noexpand\replacestringx{\string\'}{{\string\'}}%
}
\def\nextStrMarkEmpty #1#2#3#4{}
\def\getStrStyle #1{\ifcase #1strMLStylePy\or strMLStylePy\else strSLStylePy\fi}
\def\identity #1{#1} %% Odstraní obklopujici {}.
\def\nextStrMark #1#2#3\nextStrMark#4#5{%
	\if#1#4\STR{\getStrStyle #1}{#2#3#5}\expandafter\nextStrMarkEmpty
		\else \expandafter\nextStrMark
	\fi {#1}{#2}\identity{#3#5}%
}
\def\markStrPrefixPy {%
	\doreplace{\n ##1\STR}{\tmp{##1}} from \strPrefListPy
	\def\tmp ##1##2##3{\n\STR{##2}{##1##3}}%
	\edef\tmpb{\tmpb}%
}

%%%% Označkování čísel.
\def\markNumbersPy {%
	%% Normální čísla.
	\replacerange{\n0}{\n}{\numbStart0##2\end}%
	\replacerange{\n1}{\n}{\numbStart1##2\end}%
	\replacerange{\n2}{\n}{\numbStart2##2\end}%
	\replacerange{\n3}{\n}{\numbStart3##2\end}%
	\replacerange{\n4}{\n}{\numbStart4##2\end}%
	\replacerange{\n5}{\n}{\numbStart5##2\end}%
	\replacerange{\n6}{\n}{\numbStart6##2\end}%
	\replacerange{\n7}{\n}{\numbStart7##2\end}%
	\replacerange{\n8}{\n}{\numbStart8##2\end}%
	\replacerange{\n9}{\n}{\numbStart9##2\end}%
	%% Desetiná čísla.
	\replacestringx{\end\O.\numbStart}{.}%
	\replacestringx{\end\O.\n}{.\end}%
	\replacestringx{\n\O.\numbStart}{\numbStart.}%
	% %% Vědecký zápis.
	\replacerange{\end e}{\n}{e##2\end}%% Případ 2.e123
	\replacerange{\end E}{\n}{E##2\end}%%
	\doreplace{e\end\O ##1\numbStart}{e##1} from {+-}%
	\doreplace{E\end\O ##1\numbStart}{E##1} from {+-}%
	%% Komplexní čísla.
	\replacestringx{\end j}{j\end}%
	%% Vlastní provedení označkování.
	\edef\tmpb{\tmpb}%
}
\def\numbStart #1\end{\n\U{numbStylePy}{#1}\n}

%%%% Seznam klíčových slov
%%%% Pozn: bez def a class. Ty jsou zpracovány zvlášť.
\def\kwListPy {%
	{and}{as}{assert}{async}{await}{break}{continue}{del}{elif}{else}{except}%
	{exec}{False}{finally}{for}{from}{global}{if}{import}{in}{is}{lambda}{None}%
	{nonlocal}{not}{or}{pass}{print}{raise}{return}{True}{try}{while}{with}%
	{yield}}
%%%% Seznam operátorů.
\edef\opListPy {+-*//=<>()[]:,.;!\percent
	\string}\string{\string@\string&\string|\string~\string^}
%%%% Seznam prefixů stringových literálů.
\def\strPrefListPy {uUrRbB{br}{Br}{bR}{BR}{rb}{rB}{Rb}{RB}}

%%%% Dodatečné defince barev.
\def\LightBlue {\setcmykcolor{1 0.43 0 0}}
\def\Orange    {\setcmykcolor{0 0.64 1 0}}

%%%% Styly formátování jednotlivých elementů.
\def\setStylePy #1{%
	\expandafter\ifx\csname stylePy#1\endcsname\relax
		\PySyntaxWarning{Style with the name '#1' not found.}%
	\fi \csname stylePy#1\endcsname
}
\def\stylePyDEFAULT {%
	%% Standradní styl zvýraznění.
	\let\commentStylePy=\Green
	\let\opStylePy=\Blue
	\let\kwStylePy=\LightBlue
	\let\clsNameStylePy=\Black
	\let\defNameStylePy=\Magenta
	\let\numbStylePy=\Red
	\let\identStylePy=\Black
	\let\strMLStylePy=\Orange
	\let\decorStylePy=\Magenta
	\def\strSLStylePy{\def\ {\char32}\Grey}%
}
\def\stylePyIDLE {%
	%% Styl zvýraznění z editoru IDLE. (obsažen v instalaci Pythonu)
	\let\commentStylePy=\Red
	\let\opStylePy=\Black
	\let\kwStylePy=\Orange
	\let\clsNameStylePy=\LightBlue
	\let\defNameStylePy=\LightBlue
	\let\numbStylePy=\Black
	\let\identStylePy=\Black
	\let\strMLStylePy=\Green
	\let\decorStylePy=\Black
	\def\strSLStylePy {\def\ {\char32}\Green}%
}
\stylePyDEFAULT %% Standardní styl.

%%%% Formátování elementu daným stylem.
\def\PySyntaxFormat #1#2{{%
	\expandafter\ifx\csname #1\endcsname\relax
		\PySyntaxWarning {Element formating style '#1' not found.}%
	\else
		\csname #1Hook\endcsname %% Předzpracování.
		\csname #1\endcsname     %% Nastavení stylu.
	\fi	#2% Výstup obsahu elementu.
}}

%%%% Předzpracování, abychom zabránili problémům při vnoření elementů.
\def\commentStylePyHook{\def\STR ##1##2{##2}}

\endinput

%%%%%%%%%%%%%% History of versions:

Apr.04 2016 Iniciální verze makra PySyntax.
Apr.08 2016 Makro \replacestrings nahrazeno optimalizovaným \replacestringx.
Apr.10 2016 Vložení kontrol pro používané makro OPmac.
            Přidání klíčových slov "async" a "await", které přišly s Python3.5.
            Vyřešení vnořených komentářů.
            Barvení v definicích funkcí.
Apr.11 2016 Ošetření tabulátorů, které indikují bloky.
            Úprava představení makra, aby nedocházelo k zalamování v konzoli.
Apr.16 2016 Přidání makra \replacerange a zjednodušení kódu.
            Přidání zbylých operátorů a oddělovačů dle specifikace jazyka.
            Přidání prefixových operátorů před stringy.
Apr.17 2016 Zvýraznění čísel podle specifikace jazyka (plně).
Apr.20 2016 Zavedení přepínačů stylů.
            Přidání nastavitelného rozměru tabulátorů.
Apr.24 2016 Přepracování stylů a jejich nastavení.
            Přizpůsobení stylu chyb a varování tomu co je v OPmac.
            Refaktorizace a sjednocení celého kódu.
            Přidání zvýraznění dekorátorů jazyka Python.
Apr.26 2016 Přidání háků pro předzpracování těsně před formátováním.
            Ošetření řetězců vnořených v komentářích pomocí háků.
Apr.28 2016 Odstraněný chybový případ v \replacestringx a \replacerange pomocí
            znaku '?'. Nutnost použití je stejná jako v \replacestrings.
       INFO Odevzdání ke kontrole.
Apr.30 2016 Rozšíření dekorátorů na celý řádek místo na identifikátor.

%%%%%%%%%%%%%% EOF pysyntax.tex
